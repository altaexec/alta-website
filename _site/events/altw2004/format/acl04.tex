%
% File acl04.tex
% January 07 2004
% Contact: rambow@cs.columbia.edu

\documentclass[11pt]{article}
\usepackage{acl04}
\usepackage{times}
\usepackage{latexsym}

\title{Instructions for ACL-2004 Submissions and Proceedings}

\author{Jane Doe\\
  Department of Computer Science \\
  Nonesuch State University \\
  Utopia, NS 12345 \\
  {\tt jane.doe@cs.nsu.edu} \And
  John Smith \\
  Department of Linguistics \\
  Another State University \\
  Collegetown, AS 98765 \\  
  {\tt jsmith@ling.asu.edu}}

\date{}

\begin{document}

\maketitle

\begin{abstract}
  This document contains the instructions for preparing a submission to the main
  track in ACL-2004.  The instructions for submitting a camera-ready
  manuscript for the proceedings of ACL-2004 will be published later, but
  the format will be identical to that described here.
  This document conforms
  to its own specifications, and is therefore an example of what
  your manuscript should look like.  Authors are asked to conform to
  all the directions reported in this document.
  
  A major {\em format difference} between the submitted version and the
  final, camera-ready manuscript (and probably the only one) is that {\bf
    authors' names and affiliation should not be included in the version
    submitted for review.}
\end{abstract}

\section{Introduction}

The following instructions are directed to authors of papers submitted to
the main track in ACL-2004.  The submission is electronic.  Precise
instructions for submission of camera-ready copies of the accepted papers
will follow; however, the formatting of the final version will be similar
to the submission formatting.  All authors are required to adhere to these
specifications for the submission.  (Note that the proceedings will be
printed on DIN A4 paper, and the final version {\em must}\/ be in A4
format.)

Authors from countries in which access to word-processing systems is
limited should contact the PC Walter Daelemans (walter.daelemans@ua.ac.be)
as soon as possible.

\section{Using the provided style files}

The easiest way to correctly format your paper is to use one of the
style files provided on the conference web page. For instance, \LaTeX{}
users can use the {\bf acl04.sty} style file, and Microsoft Word
users can use the {\bf acl04.dot} document template.

Figure \ref{latex-skeleton} shows a skeleton document for \LaTeX{}
users.  To use it, simply copy the text in the figure to a file and then
replace the parts flagged by {\bf TODO} as appropriate for your
document.  Figure \ref{bibtex-skeleton} shows the simple Bib\TeX{} file
used by the skeleton document.

%% By using these style files most of the requirements discussed in the
%% following sections will be automatically handled. Nonetheless, you should
%% review the sections to become familiar with the requirements.

\begin{figure}

%%........................................................................
%% start of embedded latex example
\begin{small}
\begin{verbatim}
\documentclass[11pt]{article}
\usepackage{acl04}
\usepackage{times}
\usepackage{latexsym}

\title{TODO: put title here}

\author{}

\date{}

\begin{document}
\maketitle

\begin{abstract}
TODO: put the abstract here
\end{abstract}

\section{Introduction}
TODO: add introduction; and,
perhaps cite WordNet \cite{Miller-90}

\section{TODO: add body of paper}

\section{Conclusion}
TODO: add conclusion

\section*{Acknowledgements}
TODO: add acknowledgements

\bibliographystyle{acl}
%% TODO: use base name of your .bib file
\bibliography{my-citations}     

\end{document}
\end{verbatim}
\end{small}
%% end of embedded latex example
%%........................................................................

\caption{Skeleton \LaTeX{} document illustrating use of {\bf acl04.sty} and
  {\bf acl.bst}}
\label{latex-skeleton}
\end{figure}


\begin{figure}
\begin{small}
\begin{verbatim}
@Article{Miller-90,
  author =   {G. Miller},
  title =    {Special Issue on {WordNet}},
  journal =  {International Journal of
              Lexicography},
  year =     {1990},
  volume =   {3(4)},
}
\end{verbatim}
\end{small}
\caption{Bib\TeX{} file {\bf my-citations.bib} used in skeleton document}
\label{bibtex-skeleton}
\end{figure}

\section{General Instructions}

Manuscripts must be in two-column format.  Exceptions to the two-column
format include the title, which must be centered at the top of the first
page, and any full-width figures or tables (see the guidelines in
Subsection~\ref{ssec:first}).  {\bf Use single spacing.}  See the
guidelines later regarding formatting the first page.

The maximum length of a manuscript is eight ($8$) pages for the main
conference (see Subsection~\ref{ssec:layout} for exact page layout and
Section~\ref{sec:length} for additional information on the maximum number
of pages).


\section{Format of Electronic Manuscript}
\label{sect:pdf}

For the production of the submission you must use Adobe's Portable Document
Format (PDF). This format can be generated from postscript files: on Unix
systems, you can use {\tt ps2pdf} for this purpose; under Microsoft
Windows, Adobe's Distiller can be used.  Note that some word processing
programs generate PDF which may not include all the necessary fonts (esp.
tree diagrams, symbols). When you print or create the PDF file, there is
usually an option in your printer setup to include none, all or just
non-standard fonts.  Please make sure that you select the option of
including ALL the fonts.  {\em Before sending it, test your {\/\em PDF} by
  printing it from a computer different from the one where it was created}.
Moreover, some word processor may generate very large postscript/PDF files,
where each page is rendered as an image. Such images may reproduce poorly.
In this case, try alternative ways of obtaining the postscript and/or PDF.
One way on some systems is to install a driver for a postscript printer,
send your document to the printer specifying ``Output to a file'', then
convert the file to PDF.

For reasons of uniformity, Adobe's {\bf Times Roman} font should be
used. In \LaTeX2e{} this is accomplished by putting

\begin{quote}
\begin{verbatim}
\usepackage{times}
\usepackage{latexsym}
\end{verbatim}
\end{quote}
in the preamble.

Additionally, it is of utmost importance to specify the {\bf
  A4 format} when formatting the paper.
When working with {\tt dvips}, for instance, one should specify {\tt
  -t a4}.

%Print-outs of the PDF file on US-Letter paper should be identical to the
%hardcopy version.

\subsection{Layout}
\label{ssec:layout}


Format the manuscript two columns to a page, in the manner these
instructions are formatted.  In no case should the font for the body of the
paper be smaller than 11pt, and the dimensions of the text area proper must
be 230mm x 160mm (9" by 6.3") or less.

Papers should not be submitted formatted for any other paper size.
Exceptionally, authors for whom it is \emph{impossible} to format on A4
paper may use \emph{US-letter} paper. In this case, they should keep the
text area as given above, and modify the bottom and right margins as
necessary.  

\subsection{The First Page}
\label{ssec:first}

Center the title across both columns. Do not include author names or
affiliations.  Use the two-column format only when you begin the abstract.

{\bf Title}: Place the title centered at the top of the first page, in
a 15-point bold font. Long titles should be typed on two lines without
a blank line intervening. Approximately, put the title at 1in from the
top of the page.

{\bf Abstract}: Type the abstract at the beginning of the first
column.  The width of the abstract text should be smaller than the
width of the columns for the text in the body of the paper by about
0.25in on each side.  Center the word {\bf Abstract} in a 12 point
bold font above the body of the abstract. The abstract should be a
concise summary of the general thesis and conclusions of the paper.
It should be no longer than 200 words.

{\bf Text}: Begin typing the main body of the text immediately after
the abstract, observing the two-column format as shown in 
the present document. Type single spaced. 

{\bf Indent} when starting a new paragraph. For reasons of uniformity,
use Adobe's {\bf Times Roman} fonts, with 11 points for text and 
subsection headings, 12 points for section headings and 15 points for
the title. If Times Roman is unavailable, use {\bf Computer Modern
  Roman} (\LaTeX2e{}'s default; see section \ref{sect:pdf} above).
Note that the latter is about 10\% less dense than Adobe's Times Roman
font.

\subsection{Sections}

{\bf Headings}: Type and label section and subsection headings in the
style shown on the present document.  Use numbered sections (Arabic
numerals) in order to facilitate cross references. Number subsections
with the section number and the subsection number separated by a dot,
in Arabic numerals. Do not number subsubsections.

{\bf Citations}: Follow the ``Guidelines for Formatting Submissions''
to {\em Computational Linguistics\/} that appears in the first issue of
each volume, if possible.  That is, citations within the text appear
in parentheses as~\cite{Gusfield:97} or, if the author's name appears in
the text itself, as Gusfield~\shortcite{Gusfield:97}. 
Append lowercase letters to the year in cases of ambiguities.  
Treat double authors as in~\cite{Aho:72}, but write as 
in~\cite{Chandra:81} when more than two authors are involved. 
Collapse multiple citations as in~\cite{Gusfield:97,Aho:72}.

\textbf{References}: Gather the full set of references together under
the heading {\bf References}; place the section before any Appendices,
unless they contain references. Arrange the references alphabetically
by first author, rather than by order of occurrence in the text.
Provide as complete a citation as possible, using a consistent format,
such as the one for {\em Computational Linguistics\/} or the one in the 
{\em Publication Manual of the American 
Psychological Association\/}~\cite{APA:83}.  Use of full names for
authors rather than initials is preferred.  A list of abbreviations
for common computer science journals can be found in the ACM 
{\em Computing Reviews\/}~\cite{ACM:83}.

The provided \LaTeX{} and Bib\TeX{} style files roughly fit the
American Psychological Association format, allowing regular citations, 
short citations and multiple citations as described above.

{\bf Appendices}: Appendices, if any, directly follow the text and the
references (but see above).  Letter them in sequence and provide an
informative title: {\bf Appendix A. Title of Appendix}.

The \textbf{Acknowledgements} section should go as the last section immediately
before the references.  Do not number the acknowledgements section.

\subsection{Footnotes}

{\bf Footnotes}: Put footnotes at the bottom of the page. They may
be numbered or referred to by asterisks or other
symbols.\footnote{This is how a footnote should appear.} Footnotes
should be separated from the text by a line.\footnote{Note the
line separating the footnotes from the text.}

\subsection{Graphics}

{\bf Illustrations}: Place figures, tables, and photographs in the
paper near where they are first discussed, rather than at the end, if
possible.  Wide illustrations may run across both columns. Do not use
color illustrations as they may reproduce poorly.

{\bf Captions}: Provide a caption for every illustration; number each one
sequentially in the form:  ``Figure 1. Caption of the Figure.'' ``Table 1.
Caption of the Table.''  Type the captions of the figures and 
tables below the body, using 11 point text.  


\section{Length of Submission}
\label{sec:length}

Eight pages ($8$) is the maximum length of papers for submissions to the
ACL-2004 main conference. All illustrations, references, and appendices
must be accommodated within these page limits, observing the formatting
instructions given in the present document.  Papers that do not conform to
the specified length and formatting requirements are subject to be rejected
without review.



\section*{Acknowledgements}

This document has been adapted from the instructions for ACL 2003.  These
were adapted by Tom O'Hara and Dan Roth from the instructions for the
ACL-2002 proceedings by Dekang Lin and Eugene Charniak.  It was adapted in
turn from the ACL-01 proceedings by Norbert Reithinger, Giorgio Satta, and
Roberto Zamparelli. The ACL-01 instructions were elaborated from similar
documents used for previous editions of the ACL and EACL annual meetings.
Those versions were written by several people, including John Chen, Henry
S. Thompson and Donald Walker. Additional elements were taken from the
formatting instructions of the Xth {\em International Joint Conference on
  Artificial Intelligence}.


%\bibliographystyle{acl}
%\bibliography{sample}

\begin{thebibliography}{}

\bibitem[\protect\citename{Aho and Ullman}1972]{Aho:72}
Alfred~V. Aho and Jeffrey~D. Ullman.
\newblock 1972.
\newblock {\em The Theory of Parsing, Translation and Compiling}, volume~1.
\newblock Prentice-{Hall}, Englewood Cliffs, NJ.

\bibitem[\protect\citename{{American Psychological Association}}1983]{APA:83}
{American Psychological Association}.
\newblock 1983.
\newblock {\em Publications Manual}.
\newblock American Psychological Association, Washington, DC.

\bibitem[\protect\citename{{Association for Computing Machinery}}1983]{ACM:83}
{Association for Computing Machinery}.
\newblock 1983.
\newblock {\em Computing Reviews}, 24(11):503--512.

\bibitem[\protect\citename{Chandra \bgroup et al.\egroup }1981]{Chandra:81}
Ashok~K. Chandra, Dexter~C. Kozen, and Larry~J. Stockmeyer.
\newblock 1981.
\newblock Alternation.
\newblock {\em Journal of the Association for Computing Machinery},
  28(1):114--133.

\bibitem[\protect\citename{Gusfield}1997]{Gusfield:97}
Dan Gusfield.
\newblock 1997.
\newblock {\em Algorithms on Strings, Trees and Sequences}.
\newblock Cambridge University Press, Cambridge, UK.

\end{thebibliography}

\end{document}
